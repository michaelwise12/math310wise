% PREAMBLE
\documentclass[12pt]{article}
\usepackage{amssymb, amsmath, amsthm}
   % libraries of additional mathematics commands  
\usepackage[paper=letterpaper, margin=1in]{geometry}
   % sets margins and space for headers
\usepackage{setspace, listings}
   % allow for adjusted line spacing and printing source code
\title{MATH 310L112\\
       Introduction to Mathematical Reasoning\\
       Assignment \#5}
\author{Michael Wise}
\date{March 4th, 2020}
% END PREAMBLE 

\begin{document}
\maketitle
%\thispagestyle{empty}
\begin{description}

\item[Exercise 6:] Suppose $a,b,c \in \mathbb{Z}$. If $a \mid b$ and $a \mid c$, then $a \mid (b + c)$.


\begin{spacing}{2}
\begin{proof}
Let $a \mid b$ and $a \mid c$ for $a,b,c \in \mathbb{Z}$. We can say $b=ka$ and $c=la$ for some $k,l \in \mathbb{Z}$. Then, $b + c = ka + la = a(k+l)$. We have just shown that their sum is a multiple of $a$ and thus, $a \mid (b + c)$.
\end{proof}
\end{spacing}
\vspace{.001in}
\item[Exercise 14:] If $n \in \mathbb{Z}$, then $5n^2 + 3n + 7$ is odd. (Try cases.)

\begin{spacing}{2}
\begin{proof}
Suppose $n \in \mathbb{Z}$. Therefore, $n$ can be even or odd. Then either $n=2a$ or $n=2a+1$ for some $a \in \mathbb{Z}$. Let's consider both of these cases.
\newline
Case 1. $n=2a$ for some $a \in \mathbb{Z}$. Then
\begin{align*}
5n^2 + 3n + 7 &= 5(2a)^2 + 3(2a) + 7 \\
&= 20a^2 + 6a + 6 + 1 \\
&= 2(10a^2 + 3a + 3) + 1.
\end{align*}
Since $(10a^2 + 3a + 3)$ is just an integer, $5n^2 + 3n + 7$ is odd.
\newline
Case 2. $n=2a+1$ for some $a \in \mathbb{Z}$. Then
\begin{align*}
5n^2 + 3n + 7 &= 5(2a+1)^2 + 3(2a+1) + 7 \\
&= 5(4a^2+4a+1) + 6a + 3 + 7 \\
&= 20a^2 + 26a + 14 + 1 \\
&= 2(10a^2 + 13a + 7) + 1.
\end{align*}
Since $(10a^2 + 13a + 7)$ is just an integer, $5n^2 + 3n + 7$ is odd.
\newline
In each case we get that $5n^2 + 3n + 7$ is odd, as desired.
\end{proof}
\end{spacing} 

\item[Exercise 16:] If two integers have the same parity, then their sum is even.

\begin{spacing}{2}
\begin{proof}
Suppose $x,y \in \mathbb{Z}$ have the same parity. Therefore, either $x,y$ are both odd or $x,y$ are both even. We have two cases:
\newline
Case 1. $x,y$ are both odd. Then, $x=2a+1$ and $y=2b+1$ for some $a,b \in \mathbb{Z}$. Then
\begin{align*}
x+y &= 2a+1 + 2b+1 \\
&= 2a + 2b + 2 \\
&= 2(a + b + 1).
\end{align*}
Since $(a+b+1)$ is an integer, by definition, $x+y$ is even.
\newline
Case 2. $x,y$ are both even. Then, $x=2a$ and $y=2b$ for some $a,b \in \mathbb{Z}$. Then
\begin{align*}
x+y &= 2a + 2b \\
&= 2a + 2b \\
&= 2(a + b).
\end{align*}
Since $(a+b)$ is an integer, by definition, $x+y$ is even.
\newline
Because $x+y$ is even in both cases, we are done.
\end{proof}
\end{spacing} 
\end{description}
%%%%%%%%%%%%%%%%%%%%%%%%%%%%%%%%%%%%%%%%%%%%%%%%%%%%%%%%%%%%%%
% The commands in this section print the source code starting
% on a new page. Comment out or delete if you do not want to 
% include the source code in your document.
%
\newpage
\lstset{
   basicstyle=\footnotesize\ttfamily,
   breaklines=true,
   language=[LaTeX]{TeX}
   }
\lstinputlisting{Assignment5.tex} % Change to correct filename
%
%%%%%%%%%%%%%%%%%%%%%%%%%%%%%%%%%%%%%%%%%%%%%%%%%%%%%%%%%%%%%%
\end{document}