% PREAMBLE
\documentclass[12pt]{article}
\usepackage{amssymb, amsmath, amsthm}
   % libraries of additional mathematics commands  
\usepackage[paper=letterpaper, margin=1in]{geometry}
   % sets margins and space for headers
\usepackage{setspace, listings}
   % allow for adjusted line spacing and printing source code

\title{MATH 310L112\\
       Introduction to Mathematical Reasoning\\
       Fun With \LaTeX}
\author{Michael Wise}
\date{February 19th, 2020}
% END PREAMBLE 

\begin{document}
\maketitle
%\thispagestyle{empty}
\begin{description}

\item[Exercise 42.1:] Give an example of a set $S$ such that $S \in \mathcal{P}(\mathbb{N})$.

\begin{spacing}{2}
\textit{Solution.} We want a set $S$ that is in the power set of $\mathbb{N}$. In other words, we want a set that is a subset of $\mathbb{N}$. One such set is $S = \{1,2\}$.
\end{spacing}
  
\item[Exercise 42.6:] Let $A$ and $B$ be sets. Prove that $A \cap B \subseteq B$.

\begin{spacing}{2}
\begin{proof}
Let $ x \in A \cap B$. By definition of set intersection we have that $x \in A$ and $x \in B$. Since $x \in A \cap B$ implies $x \in B$, we conclude that $A \cap B \subseteq B$.
\end{proof}
\end{spacing} 

\item[Exercise 42.9:] Let $ x \in \mathbb{Z}$. If $5x - 7$ is odd, then $x$ is even.

\begin{spacing}{2}
\begin{proof}
Let $x \in \mathbb{Z}$ and assume that $x$ is odd. Then $x=2y+1$, where $y \in \mathbb{Z}$. Therefore, \[5x-7=5(2y+1)-7=10y-2=2(5y-1).\]
Since $5y-1$ is an integer, $5x-7$ is even.
\end{proof}
\end{spacing} 

\item[Exercise 42.12:] Let $ x \in \mathbb{Z}$. If $3 \nmid (x^2-1)$, then $3 \mid x$.

\begin{spacing}{2}
\begin{proof}
Let $x \in \mathbb{Z}$ and assume that $3 \nmid x$. Then either $x=3q+1$ for some integer $q$, or $x=3q+2$ for some integer $q$. We consider these two cases.
\newline
Case 1. $x=3q+1$ for some integer $q$. Then
\begin{align*}
x^2-1 &= (3q+1)^2-1 \\
&= (9q^2+6q+1)-1 \\
&= 9q^2+6q \\
&= 3(3q^2+2q).
\end{align*}
Since $(3q^2+2q)$ is an integer, $3 \mid (x^2-1)$.

\newline
Case 2. $x=3q+2$ for some integer $q$. Then
\begin{align*}
x^2-1 &= (3q+2)^2-1 \\
&= (9q^2+12q+4)-1 \\
&= 9q^2+12q+3 \\
&= 3(3q^2+4q+1).
\end{align*}
Since $(3q^2+4q+1)$ is an integer, $3 \mid (x^2-1)$.
\end{proof}
\end{spacing} 
\end{description}

%%%%%%%%%%%%%%%%%%%%%%%%%%%%%%%%%%%%%%%%%%%%%%%%%%%%%%%%%%%%%%
% The commands in this section print the source code starting
% on a new page. Comment out or delete if you do not want to 
% include the source code in your document.
%
\newpage
\lstset{
   basicstyle=\footnotesize\ttfamily,
   breaklines=true,
   language=[LaTeX]{TeX}
   }
\lstinputlisting{HW template.tex} % Change to correct filename
%
%%%%%%%%%%%%%%%%%%%%%%%%%%%%%%%%%%%%%%%%%%%%%%%%%%%%%%%%%%%%%%

\end{document}