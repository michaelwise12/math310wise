% PREAMBLE
\documentclass[12pt]{article}
\usepackage{amssymb, amsmath, amsthm}
   % libraries of additional mathematics commands  
\usepackage[paper=letterpaper, margin=1in]{geometry}
   % sets margins and space for headers
\usepackage{setspace, listings}
   % allow for adjusted line spacing and printing source code
\title{MATH 310L112\\
       Introduction to Mathematical Reasoning\\
       Test \#2}
\author{Michael Wise}
\date{April 15th, 2020}
% END PREAMBLE 

\begin{document}
\maketitle
%\thispagestyle{empty}
I hereby affirm that I did not and will not discuss the contents of this test using any means of communication, with any person other than Prof. Krog, between Tuesday, April 14, 2020 at 5:00 p.m. EDT (UTC -4:00) and Wednesday, April 15, 2020 at 5:00 p.m. EDT (UTC -4:00).

\begin{description}
\item[Problem 1:] Suppose $x,y \in \mathbb{R}$. Prove that if $3x^3 + 21y = 3y^3 + 21x$, then $x = y$ or $x^2 + xy +y^2 = 7$.
\begin{spacing}{2}
\begin{proof}
Assume that $3x^3 + 21y = 3y^3 + 21x$ for some $x,y \in \mathbb{R}$. Then $3x^3 -3y^3 = 21x - 21y$. Factoring on both sides gives $3 (x-y)(x^2 + xy+ y^2) = 21(x-y)$. Dividing by $3$ leaves us with $(x-y)(x^2 + xy+ y^2) = 7(x-y)$. We consider two cases.
\newline
\textbf{Case 1.} If $x-y=0$, then $x-y+y=0+y$ by adding $y$ to both sides. Thus $x=y$.
\newline
\textbf{Case 2.} If $x-y \neq 0$, then we can divide both sides of $(x-y)(x^2 + xy+ y^2) = 7(x-y)$ by $x-y$ to get $x^2 + xy +y^2 = 7$.
\newline
Therefore $x=y$ or $x^2 + xy +y^2 = 7$.
\end{proof}
\end{spacing}
\item[Problem 2:] Let $n \in \mathbb{Z}.$ Prove that if $n^2 + 2n - 4 \not \equiv 1\pmod{5}$, then $5 \nmid n$.
\begin{spacing}{2}
\begin{proof}
(Contrapositive) Let $n \in \mathbb{Z}$ and suppose that $5 \mid n$. Then $n = 5k$ for some $k \in \mathbb{Z}$. Observe that
\begin{align*}
n^2 + 2n -5 &= 25k^2 + 10k - 5 \\
&= 5(5k^2 + 2k - 1),
\end{align*}
where $5k^2 + 2k - 1 \in \mathbb{Z}$. Therefore $5 \mid (n^2 + 2n - 5)$. We can also write this as $5 \mid ((n^2 + 2n - 4) - 1)$. Thus, by definition, $n^2 + 2n - 4 \equiv 1\pmod{5}$.
\end{proof}
\end{spacing} 

\item[Problem 3:] Prove that there exists an $n \in \mathbb{N}$ such that $7 \mid (3^n - 1)$.

\begin{spacing}{2}
\begin{proof}
Suppose $7 \mid (3^n - 1)$. Therefore $3^n -1 = 7k$ for some $k \in \mathbb{Z}$. If we can find a natural number $n$ for which $3^n -1$ results in a multiple of $7$, then we are done. 
\newline
Let $n = 6$.  Then
\begin{align*}
3^n - 1 = 3^6 -1 = 729 - 1 = 728 = 7(104).
\end{align*}
We know with certainty that $6 \in \mathbb{N}$. Thus, the proof is complete.
\end{proof}
\end{spacing} 
\item[Problem 5:] The inequality $\frac{1}{2^x} \leq \frac{1}{x+1}$ is true for all positive real numbers $x$.
\begin{spacing}{2}
\begin{proof}[Disproof]
This statement is \textbf{false}. For a counterexample, consider when $x = \frac{1}{2}$. Plugging into the inequality gives us  $\frac{1}{2^\frac{1}{2}} \leq \frac{1}{\frac{1}{2}+1}$. Simplifying the denominators gives us $\frac{1}{\sqrt{2}} \leq \frac{1}{\frac{3}{2}}$. Then we get $\frac{\sqrt{2}}{2} \leq \frac{2}{3}$. Converting to decimals shows $\frac{\sqrt{2}}{2} \approx 0.707 \not \leq 0.\overline{666}$. In this example $\frac{1}{2^x} > \frac{1}{x+1}$, making the statement false.
\end{proof}
\end{spacing} 
\end{description}
%%%%%%%%%%%%%%%%%%%%%%%%%%%%%%%%%%%%%%%%%%%%%%%%%%%%%%%%%%%%%%
% The commands in this section print the source code starting
% on a new page. Comment out or delete if you do not want to 
% include the source code in your document.
%
\newpage
\lstset{
   basicstyle=\footnotesize\ttfamily,
   breaklines=true,
   language=[LaTeX]{TeX}
   }
\lstinputlisting{Test2.tex} % Change to correct filename
%
%%%%%%%%%%%%%%%%%%%%%%%%%%%%%%%%%%%%%%%%%%%%%%%%%%%%%%%%%%%%%%
\end{document}