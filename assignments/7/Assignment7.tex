% PREAMBLE
\documentclass[12pt]{article}
\usepackage{amssymb, amsmath, amsthm}
   % libraries of additional mathematics commands  
\usepackage[paper=letterpaper, margin=1in]{geometry}
   % sets margins and space for headers
\usepackage{setspace, listings}
   % allow for adjusted line spacing and printing source code
\title{MATH 310L112\\
       Introduction to Mathematical Reasoning\\
       Assignment \#7}
\author{Michael Wise}
\date{April 5th, 2020}
% END PREAMBLE 

\begin{document}
\maketitle
%\thispagestyle{empty}
\begin{description}

\item[Exercise 12:] Suppose $a \in \mathbb{Z}$. If $a^2$ is not divisible by $4$, then $a$ is odd.

\begin{spacing}{2}
\begin{proof}
(By Contrapositive) Suppose $a$ is even integer. By definition, this means $a=2k$ for some $k \in \mathbb{Z}$. Therefore, $a^2 = 4k^2$. Since $k^2 \in \mathbb{Z}$, then $a^2$ is divisible by $4$.
\end{proof}
\end{spacing}
\item[Exercise 20:] If $a \in \mathbb{Z}$ and $a \equiv 1\pmod{5}$, then $a^2 \equiv 1\pmod{5}$.

\begin{spacing}{2}
\begin{proof}
(Direct) Suppose $a \in \mathbb{Z}$ and $a \equiv 1\pmod{5}$. Then by definition, $5 \mid (a - 1)$. Thus, $a-1=5k$ for some $k \in \mathbb{Z}$. Moving $1$ to the other side gets us $a = 5k + 1$. Then
\begin{align*}
a^2 &= (5k+1)^2 \\
&= 25k^2 + 10k + 1 \\
&= 5(5k^2 + 2k) + 1.
\end{align*}
Let $l = 5k^2 + 2k \in \mathbb{Z}.$ Therefore, $a^2 = 5l + 1$ which means $5 \mid (a^2 - 1)$. By definition, we have shown that $a^2 \equiv 1\pmod{5}$.
\end{proof}
\end{spacing} 

\item[Exercise 32:] If $a \equiv b\pmod{n}$, then $a$ and $b$ have the same remainder when divided by $n$.

\begin{spacing}{2}
\begin{proof}
(Direct) Suppose $a,b,n \in \mathbb{Z}$ and $a \equiv b\pmod{n}$. This means that $n \mid (a-b)$. Then $a = nk + b$ for some $k \in \mathbb{Z}$. By the division algorithm, there exists unique integers $q$ and $r$ with $b = qn +r$ and $0 \leq r < n$. Using substitution we get
\begin{align*}
a &= nk + b \\
&= nk + qn + r \\
&= n(k+q) + r.
\end{align*}
Therefore, $r$ is also the remainder of $a$ divided by $n$.
\end{proof}
\end{spacing} 
\section*{Chapter 6}
\item[Exercise 10:] There exist no integers $a$ and $b$ for which $21a+30b=1$.
\begin{spacing}{2}
\begin{proof}
Suppose for the sake of contradiction that there are integers $a$ and $b$ that exist for which $21a+30b=1$. If this is the case, then $3(7a+10b)=1 \implies 7a + 10b = \frac{1}{3}$. Since $a,b \in \mathbb{Z}$, then $7a+10b \in \mathbb{Z}$ when following multiplicative and additive integer properties. However, we have shown $7a + 10b = \frac{1}{3} \notin \mathbb{Z}$. This is a contradiction, and thus the original proposition must be true.
\end{proof}
\end{spacing} 
\end{description}
%%%%%%%%%%%%%%%%%%%%%%%%%%%%%%%%%%%%%%%%%%%%%%%%%%%%%%%%%%%%%%
% The commands in this section print the source code starting
% on a new page. Comment out or delete if you do not want to 
% include the source code in your document.
%
\newpage
\lstset{
   basicstyle=\footnotesize\ttfamily,
   breaklines=true,
   language=[LaTeX]{TeX}
   }
\lstinputlisting{Assignment7.tex} % Change to correct filename
%
%%%%%%%%%%%%%%%%%%%%%%%%%%%%%%%%%%%%%%%%%%%%%%%%%%%%%%%%%%%%%%
\end{document}