% PREAMBLE
\documentclass[12pt]{article}
\usepackage{amssymb, amsmath, amsthm}
   % libraries of additional mathematics commands  
\usepackage[paper=letterpaper, margin=1in]{geometry}
   % sets margins and space for headers
\usepackage{setspace, listings}
   % allow for adjusted line spacing and printing source code
\title{MATH 310L112\\
       Introduction to Mathematical Reasoning\\
       Presentation \#1}
\author{Michael Wise}
\date{March 11th, 2020}
% END PREAMBLE 

\begin{document}
\maketitle
%\thispagestyle{empty}
\begin{description}

\item[Problem 3:] If $a$ and $c$ are odd integers, then $ab + bc$ is even for every integer $b$.

\begin{spacing}{2}
\begin{proof}
Suppose that $a$ and $c$ are odd integers, and $b \in \mathbb{Z}$. By definition, this means $a=2k+1$ and $c=2l+1$ for some $k,l \in \mathbb{Z}$. Then
\begin{align*}
ab + bc &= b(a+c) \\
&= b(2k+1 + 2l+1) \\
&= b(2k+2l+2) \\
&= 2b(k+l+1).
\end{align*}
Since $b(k+l+1)$ is an integer, $ab+bc$ is even.
\end{proof}
\end{spacing}
\end{description}
%%%%%%%%%%%%%%%%%%%%%%%%%%%%%%%%%%%%%%%%%%%%%%%%%%%%%%%%%%%%%%
% The commands in this section print the source code starting
% on a new page. Comment out or delete if you do not want to 
% include the source code in your document.
%
\newpage
\lstset{
   basicstyle=\footnotesize\ttfamily,
   breaklines=true,
   language=[LaTeX]{TeX}
   }
\lstinputlisting{Presentation1.tex} % Change to correct filename
%
%%%%%%%%%%%%%%%%%%%%%%%%%%%%%%%%%%%%%%%%%%%%%%%%%%%%%%%%%%%%%%
\end{document}