% PREAMBLE
\documentclass[12pt]{article}
\usepackage{amssymb, amsmath, amsthm}
   % libraries of additional mathematics commands  
\usepackage[paper=letterpaper, margin=1in]{geometry}
   % sets margins and space for headers
\usepackage{setspace, listings}
   % allow for adjusted line spacing and printing source code
\title{MATH 310L112\\
       Introduction to Mathematical Reasoning\\
       Assignment \#9}
\author{Michael Wise}
\date{April 19th, 2020}
% END PREAMBLE 

\begin{document}
\maketitle
%\thispagestyle{empty}
\begin{description}
\item[Exercise 18:] Suppose $A_1, A_2, \ldots A_n$ are sets in some universal set $U$, and $n \geq 2$. Prove that $\overline{A_1 \cup A_2 \cup \cdots \cup A_n} = \overline{A_1} \cap \overline{A_2} \cap \cdots \cap \overline{A_n}$.
\begin{spacing}{2}
\begin{proof}
 We will prove this with mathematical induction. 
\begin{enumerate}
    \item[(1)] Suppose that $n=2$. We know that $\overline{A_1 \cup A_2} = \overline{A_1} \cap \overline{A_2}$ by DeMorgan's law.
    \item[(2)] We must now prove that $S_k \Rightarrow S_{k+1}$ for any $k \geq 2$. That is, we need to show that if $\overline{A_1 \cup A_2 \cup \cdots \cup A_k} = \overline{A_1} \cap \overline{A_2} \cap \cdots \cap \overline{A_k}$, then $\overline{A_1 \cup A_2 \cup \cdots \cup A_{k+1}} = \overline{A_1} \cap \overline{A_2} \cap \cdots \cap \overline{A_{k+1}}$. We will use direct proof. Suppose that $\overline{A_1 \cup A_2 \cup \cdots \cup A_k} = \overline{A_1} \cap \overline{A_2} \cap \cdots \cap \overline{A_k}$. Then by DeMorgan's law
    \begin{align*}
    \overline{A_1 \cup A_2 \cup \cdots \cup A_{k} \cup A_{k+1}} &= \overline{(A_1 \cup A_2 \cup \cdots \cup A_{k}) \cup A_{k+1}} \\
    &= \overline{A_1 \cup A_2 \cup \cdots \cup A_{k}} \cap \overline{A_{k+1}} \\
    &= \overline{A_1} \cap \overline{A_2} \cap \cdots \cap \overline{A_k} \cap \overline{A_{k+1}}.
    \end{align*}
    \end{enumerate}
Since it is also true for $n = k+1$, it follows by induction that $\overline{A_1 \cup A_2 \cup \cdots \cup A_n} = \overline{A_1} \cap \overline{A_2} \cap \cdots \cap \overline{A_n}$ for $n \geq 2$.
\end{proof}

\end{spacing}
\item[Exercise 34:] Prove that $3^1 + 3^2 + 3^3 + 3^4 + \cdots + 3^n = \dfrac{3^{n+1}-3}{2}$ for every $n \in \mathbb{N}$.
\begin{spacing}{2}
\begin{proof}
We will prove this with mathematical induction.
\begin{enumerate}
    \item[(1)] Observe that if $n=1$, this statement is $3^1 = \dfrac{3^{1+1}-3}{2}$, and this simplifies to $3 = \dfrac{6}{2}$, which is obviously true.
    \item[(2)] Consider any integer $k \geq 1$. We need to show that $3^1 + 3^2 + \cdots + 3^k = \dfrac{3^{k+1}-3}{2}$ implies $3^1 + 3^2 + \cdots + 3^{k+1} = \dfrac{3^{(k+1)+1}-3}{2}$. We use direct proof. Suppose that $3^1 + 3^2 + \cdots + 3^k = \dfrac{3^{k+1}-3}{2}$. Then
    \begin{align*}
    3^1 + 3^2 + \cdots + 3^k + 3^{k+1} &= (3^1 + 3^2 + \cdots + 3^k) + 3^{k+1} \\
    &= \dfrac{3^{k+1}-3}{2} + 3^{k+1} \\
    &= \dfrac{3^{k+1}-3 + 2(3^{k+1})}{2} \\
    &= \dfrac{3(3^{k+1})-3}{2} \\
    &= \dfrac{3^{k+2}-3}{2}.
    \end{align*}
    \end{enumerate}
Therefore, by induction $3^1 + 3^2 + 3^3 + 3^4 + \cdots + 3^n = \dfrac{3^{n+1}-3}{2}$ for every $n \in \mathbb{N}$.
\end{proof}
\end{spacing} 
\end{description}
%%%%%%%%%%%%%%%%%%%%%%%%%%%%%%%%%%%%%%%%%%%%%%%%%%%%%%%%%%%%%%
% The commands in this section print the source code starting
% on a new page. Comment out or delete if you do not want to 
% include the source code in your document.
%
\newpage
\lstset{
   basicstyle=\footnotesize\ttfamily,
   breaklines=true,
   language=[LaTeX]{TeX}
   }
\lstinputlisting{Assignment9.tex} % Change to correct filename
%
%%%%%%%%%%%%%%%%%%%%%%%%%%%%%%%%%%%%%%%%%%%%%%%%%%%%%%%%%%%%%%
\end{document}